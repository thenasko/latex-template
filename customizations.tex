%%%%%%%%%%%%%%%%%%%%%%%%%%%%%%%%
%%                            %%
%%    LaTeX Customizations    %%
%%      Atanas Atanasov       %%
%%                            %%
%%%%%%%%%%%%%%%%%%%%%%%%%%%%%%%%

%% Customizations styles
% \newcommand{\boldsymbols}{}{}
\newcommand{\showtodos}{}


%% Packages
\usepackage{amsmath, amsthm, amssymb}
% \usepackage{mathabx} %% Use only if necessary (required for \widecheck)
\usepackage{hyperref}
\usepackage{array}
% \usepackage{fullpage}
\usepackage{ifthen}
\usepackage{ifpdf}
\ifpdf
  \usepackage[
    pdftex,
    letterpaper,
    includeheadfoot,
    centering=true,
    margin=1in,
    bindingoffset=0in
  ]{geometry}
  \usepackage[pdftex,final]{graphicx}  
\else
  \usepackage[
    dvips,
    letterpaper,
    includeheadfoot,
    centering=true,
    margin=1in,
    bindingoffset=0in
  ]{geometry}
  \usepackage[dvips,final]{graphicx}
\fi
% \usepackage{psfragx}
\usepackage{indentfirst}
\usepackage{suffix}
\usepackage{float}
\usepackage{enumerate}
% \usepackage{ifthen}
% \usepackage{listings}
% \usepackage{hyperref}
\usepackage[all,2cell,knot]{xy}
\UseAllTwocells
\SilentMatrices
\SelectTips{cm}{}
\xyoption{arc}
\xyoption{pdf}
% \usepackage[mathscr]{eucal}
\usepackage{bm}
\usepackage{mathrsfs}
\usepackage{booktabs}
% \usepackage{leftidx}
\usepackage[vcentermath]{youngtab}
\newcommand{\ysp}{\textrm{}}
% \usepackage{mathdots}
% \usepackage[inline]{showlabels}
\usepackage[final]{showlabels}
\usepackage[capitalize,nameinlink,noabbrev]{cleveref}
% \usepackage{sagetex}


%% Fonts

%% Palatino fonts
% \usepackage[sc]{mathpazo}
% \linespread{1.05}         % Palatino needs more leading (space between lines)
% \usepackage[T1]{fontenc}

%% MinionPro fonts
\IfFileExists{MinionPro.sty}{\usepackage{MinionPro}}{}


%% Miscellaneous

% \replacecommand is similar to \providecommand but it initializes the
% command as necessary even if a previous definition exists.
\newcommand{\replacecommand}[2]{\providecommand{#1}{}\renewcommand{#1}{#2}}

%% Environments

\theoremstyle{plain}
% \numberwithin{equation}{section}
\newtheorem{theorem}[equation]{Theorem}
\newtheorem{proposition}[equation]{Proposition}
\newtheorem{lemma}[equation]{Lemma}
\newtheorem{corollary}[equation]{Corollary}
% \newtheorem{claim}[equation]{Claim}
\newtheorem{conjecture}[equation]{Conjecture}
\newtheorem{observation}[equation]{Observation}

\theoremstyle{definition}
\newtheorem{definition}[equation]{Definition}
\newtheorem{exercise}[equation]{Exercise}
\newtheorem{question}[equation]{Question}
% \newtheorem{problem}[equation]{Problem}
\newtheorem{construction}[equation]{Construction}

\theoremstyle{remark}
\newtheorem{example}[equation]{Example}
% \newtheorem{hint}[equation]{Hint}
\newtheorem{remark}[equation]{Remark}
\newtheorem{fact}[equation]{Fact}
\newtheorem{warning}[equation]{Warning}

%% Names theorem environment
\newtheorem*{namedtheorem}{\theoremname}
\newcommand{\theoremname}{}
\newenvironment{named}[1]{\renewcommand{\theoremname}{#1}\begin{namedtheorem}}{\end{namedtheorem}}
  
%% Problem-set commands (also useful for worksheets, exams, etc.)
% \newcounter{cnt_problem}
% \setcounter{cnt_problem}{1}
% \newcommand\problem[1][Problem]{\vspace{0.15in} \noindent {\hskip \labelsep {\bfseries #1 \arabic{cnt_problem}}} \addtocounter{cnt_problem}{1}}
% \WithSuffix\newcommand\problem*[1]{\vspace{0.15in} \noindent {\hskip \labelsep {\bfseries #1}}}

\newenvironment{claim}[1][Claim.]{\begin{trivlist}
\item[\hskip \labelsep {\bfseries #1}]}{\end{trivlist}}
\newenvironment{claim-mild}[1][Claim.]{{\bfseries #1} \hskip 0.2\labelsep}{}


% \floatstyle{plain}
% \newfloat{figure}{thp}{}
% \floatname{figure}{Figure}
% \numberwithin{figure}{section}

%% Enumerate/itemize environments
\newenvironment{itemize_packed}{
\\[-20pt]
\begin{itemize}
  \setlength{\itemsep}{2pt}
  \setlength{\parskip}{0pt}
  \setlength{\parsep}{0pt}
}{\end{itemize}}

\newenvironment{enumerate_packed}[1][1.]{
\\[-20pt]
\begin{enumerate}[#1]
  \setlength{\itemsep}{2pt}
  \setlength{\parskip}{0pt}
  \setlength{\parsep}{0pt}
}{\end{enumerate}}

\newenvironment{enumeratea}
{\begin{enumerate}[\upshape (a)]}
{\end{enumerate}}

\newenvironment{enumeratei}
{\begin{enumerate}[\upshape (i)]}
{\end{enumerate}}

\newcommand{\topcell}[1]{\vtop{\null\hbox{#1}}}

%% TODO command: renders only of \showtodos is defined
\ifthenelse{\isundefined{\showtodos}}{
  \newcommand{\todo}[1]{}
}{
  \newcommand{\todo}[1]{#1}
}


%% Special font symbols

%% Bold letters: blackboard or regular bold font depending on the
%% \boldsymbols command.
\ifthenelse{\isundefined{\boldsymbols}}{
  \newcommand{\mathbbf}[1]{\mathbb{#1}}
}{
  \newcommand{\mathbbf}[1]{\mathbf{#1}}
}

\newcommand{\Abb}{\ensuremath{\mathbbf{A}}}
\newcommand{\Cbb}{\ensuremath{\mathbbf{C}}}
\newcommand{\Fbb}{\ensuremath{\mathbbf{F}}}
\newcommand{\Gbb}{\ensuremath{\mathbbf{G}}}
\newcommand{\Hbb}{\ensuremath{\mathbbf{H}}}
\newcommand{\Ibb}{\ensuremath{\mathbbf{I}}}
\newcommand{\Nbb}{\ensuremath{\mathbbf{N}}}
\newcommand{\Lbb}{\ensuremath{\mathbbf{L}}}
\newcommand{\Obb}{\ensuremath{\mathbbf{O}}}
\newcommand{\Pbb}{\ensuremath{\mathbbf{P}}}
\newcommand{\Qbb}{\ensuremath{\mathbbf{Q}}}
\newcommand{\Rbb}{\ensuremath{\mathbbf{R}}}
\newcommand{\Tbb}{\ensuremath{\mathbbf{T}}}
\newcommand{\Vbb}{\ensuremath{\mathbbf{V}}}
\newcommand{\Zbb}{\ensuremath{\mathbbf{Z}}}

%% Calligraphic letters
\newcommand{\Ac}{\ensuremath{\mathcal{A}}}
\newcommand{\Bc}{\ensuremath{\mathcal{B}}}
\newcommand{\Cc}{\ensuremath{\mathcal{C}}}
\newcommand{\Dc}{\ensuremath{\mathcal{D}}}
\newcommand{\Ec}{\ensuremath{\mathcal{E}}}
\newcommand{\Fc}{\ensuremath{\mathcal{F}}}
\newcommand{\Gc}{\ensuremath{\mathcal{G}}}
\newcommand{\Hc}{\ensuremath{\mathcal{H}}}
\newcommand{\Ic}{\ensuremath{\mathcal{I}}}
\newcommand{\Jc}{\ensuremath{\mathcal{J}}}
\newcommand{\Kc}{\ensuremath{\mathcal{K}}}
\newcommand{\Lc}{\ensuremath{\mathcal{L}}}
\replacecommand{\Mc}{\ensuremath{\mathcal{M}}}
\newcommand{\Nc}{\ensuremath{\mathcal{N}}}
\newcommand{\Oc}{\ensuremath{\mathcal{O}}}
\newcommand{\Pc}{\ensuremath{\mathcal{P}}}
\newcommand{\Qc}{\ensuremath{\mathcal{Q}}}
\newcommand{\Rc}{\ensuremath{\mathcal{R}}}
\newcommand{\Sc}{\ensuremath{\mathcal{S}}}
\newcommand{\Tc}{\ensuremath{\mathcal{T}}}
\newcommand{\Uc}{\ensuremath{\mathcal{U}}}
\newcommand{\Vc}{\ensuremath{\mathcal{V}}}
\newcommand{\Wc}{\ensuremath{\mathcal{W}}}
\newcommand{\Xc}{\ensuremath{\mathcal{X}}}

%% Script letters
\newcommand{\As}{\ensuremath{\mathscr{A}}}
\newcommand{\Bs}{\ensuremath{\mathscr{B}}}
\newcommand{\Cs}{\ensuremath{\mathscr{C}}}
\newcommand{\Ds}{\ensuremath{\mathscr{D}}}
\newcommand{\Fs}{\ensuremath{\mathscr{F}}}
\newcommand{\Gs}{\ensuremath{\mathscr{G}}}
\newcommand{\Hs}{\ensuremath{\mathscr{H}}}
\newcommand{\Ks}{\ensuremath{\mathscr{K}}}
\newcommand{\Ms}{\ensuremath{\mathscr{M}}}
\newcommand{\Ss}{\ensuremath{\mathscr{S}}}
\newcommand{\Ts}{\ensuremath{\mathscr{T}}}
\newcommand{\Ws}{\ensuremath{\mathscr{W}}}
% \newcommand{\Xs}{\ensuremath{\mathscr{X}}}
\newcommand{\Zs}{\ensuremath{\mathscr{Z}}}

%% Fraktur letters
\renewcommand{\a}{\ensuremath{\mathfrak{a}}}
\renewcommand{\b}{\ensuremath{\mathfrak{b}}}
\renewcommand{\c}{\ensuremath{\mathfrak{c}}}
\newcommand{\g}{\ensuremath{\mathfrak{g}}}
% \newcommand{\h}{\ensuremath{\mathfrak{h}}}
\newcommand{\m}{\ensuremath{\mathfrak{m}}}
\newcommand{\n}{\ensuremath{\mathfrak{n}}}
\renewcommand{\o}{\ensuremath{\mathfrak{o}}}
\newcommand{\p}{\ensuremath{\mathfrak{p}}}
\newcommand{\q}{\ensuremath{\mathfrak{q}}}
\renewcommand{\r}{\ensuremath{\mathfrak{r}}}
\newcommand{\s}{\ensuremath{\mathfrak{s}}}
\renewcommand{\t}{\ensuremath{\mathfrak{t}}}
\renewcommand{\u}{\ensuremath{\mathfrak{u}}}

%% Miscellaneous
\newcommand{\Cch}{\ensuremath{\check{C}}}
\newcommand{\Hch}{\ensuremath{\check{H}}}
\renewcommand{\d}{\ensuremath{\partial}}


%% Operators
\replacecommand{\Im}{\mathop{\textrm{Im}}\nolimits}
\replacecommand{\Re}{\mathop{\textrm{Re}}\nolimits}
\replacecommand{\H}{\mathop{\textrm{H}}\nolimits}
\replacecommand{\h}{\mathop{\textrm{h}}\nolimits}
\DeclareMathOperator{\pr}{pr}
\DeclareMathOperator{\tr}{tr}
\DeclareMathOperator{\Tr}{Tr}
\DeclareMathOperator{\Hom}{Hom}
\DeclareMathOperator{\Aut}{Aut}
\DeclareMathOperator{\Inn}{Inn}
\DeclareMathOperator{\Out}{Out}
\DeclareMathOperator{\Ker}{Ker}
\DeclareMathOperator{\Coker}{Coker}
\DeclareMathOperator{\Sym}{Sym}
\DeclareMathOperator{\Obj}{Obj}
\DeclareMathOperator{\Mor}{Mor}
\DeclareMathOperator{\Kom}{Kom}
\DeclareMathOperator{\Gal}{Gal}
\DeclareMathOperator{\Spec}{Spec}
\DeclareMathOperator{\mSpec}{\mathfrak{m}Spec}
\DeclareMathOperator{\Proj}{Proj}
\DeclareMathOperator{\Tor}{Tor}
\DeclareMathOperator{\Ext}{Ext}
\DeclareMathOperator{\End}{End}
\DeclareMathOperator{\Rep}{Rep}
\DeclareMathOperator{\Ind}{Ind}
\DeclareMathOperator{\Res}{Res}
\DeclareMathOperator{\Ann}{Ann}
\DeclareMathOperator{\Ass}{Ass}
\DeclareMathOperator{\Frac}{Frac}
\DeclareMathOperator{\Homc}{\mathcal{H}om}
\DeclareMathOperator{\Extc}{\mathcal{E}xt}
% \DeclareMathOperator{\cl}{cl}
\DeclareMathOperator{\cp}{cp}
\DeclareMathOperator{\id}{id}
\DeclareMathOperator{\ev}{ev}
\DeclareMathOperator{\height}{ht}
\DeclareMathOperator{\nil}{nil}
\DeclareMathOperator*{\colim}{colim}
\DeclareMathOperator{\codim}{codim}
\DeclareMathOperator{\depth}{depth}
\DeclareMathOperator{\ch}{ch}
\DeclareMathOperator{\td}{td}
\DeclareMathOperator{\characteristic}{char}
\DeclareMathOperator{\proj}{proj}
\DeclareMathOperator{\rad}{rad}
\DeclareMathOperator{\Conv}{Conv}
\DeclareMathOperator{\Hull}{Hull}
\DeclareMathOperator{\cone}{cone}
\DeclareMathOperator{\Supp}{Supp}
\DeclareMathOperator{\rank}{rank}
\DeclareMathOperator{\Rank}{Rank}
\DeclareMathOperator{\Span}{Span}
\DeclareMathOperator{\Map}{Map}
\DeclareMathOperator{\Mod}{Mod}
\DeclareMathOperator{\MCG}{MCG}
\DeclareMathOperator{\AMCG}{AMCG}
\DeclareMathOperator{\Diff}{Diff}
\DeclareMathOperator{\Der}{Der}
\DeclareMathOperator{\OutDer}{OutDer}
\DeclareMathOperator{\Cl}{Cl}
\DeclareMathOperator{\CaCl}{CaCl}
\DeclareMathOperator{\Pic}{Pic}
\DeclareMathOperator{\Jac}{Jac}
\DeclareMathOperator{\Bl}{Bl}
\DeclareMathOperator{\Fl}{Fl}
\DeclareMathOperator{\Gr}{Gr}
\DeclareMathOperator{\gr}{gr}
\DeclareMathOperator{\VF}{VF}
\DeclareMathOperator{\II}{II}
\DeclareMathOperator{\ad}{ad}
\DeclareMathOperator{\Ad}{Ad}
\DeclareMathOperator{\Adj}{Adj}
\DeclareMathOperator{\Pf}{Pf}
\DeclareMathOperator{\PD}{PD}
\DeclareMathOperator{\Vol}{Vol}
\DeclareMathOperator{\vol}{vol}
\DeclareMathOperator{\Ric}{Ric}
\DeclareMathOperator{\Lie}{Lie}
\DeclareMathOperator{\Sh}{Sh}
\DeclareMathOperator{\PSh}{PSh}
\DeclareMathOperator{\Sing}{Sing}
\DeclareMathOperator{\sign}{sign}
\DeclareMathOperator{\divisor}{div}
\DeclareMathOperator{\divergence}{div}
\DeclareMathOperator{\genus}{genus}
\DeclareMathOperator{\Hilb}{Hilb}
\DeclareMathOperator{\SL}{SL}
\DeclareMathOperator{\GL}{GL}
\DeclareMathOperator{\PGL}{PGL}
\DeclareMathOperator{\PSL}{PSL}
\renewcommand{\O}{\mathop{\textnormal{O}}\nolimits}
\DeclareMathOperator{\SO}{SO}
\DeclareMathOperator{\U}{U}
\DeclareMathOperator{\Sp}{Sp}
\DeclareMathOperator{\gl}{\mathfrak{gl}}
\DeclareMathOperator{\so}{\mathfrak{so}}
\DeclareMathOperator{\su}{\mathfrak{su}}
\renewcommand{\sl}{\mathop{\mathfrak{sl}}\nolimits}
\DeclareMathOperator{\ord}{ord}
\DeclareMathOperator{\simp}{simp}
\DeclareMathOperator{\cell}{cell}
\DeclareMathOperator{\ab}{ab}
\DeclareMathOperator{\diag}{diag}
\DeclareMathOperator{\res}{res}
\DeclareMathOperator{\interior}{int}
\DeclareMathOperator{\tors}{tors}
\DeclareMathOperator{\arccot}{arccot}
\DeclareMathOperator{\arcsec}{arcsec}
\DeclareMathOperator{\arccsc}{arccsc}
\DeclareMathOperator{\sech}{sech}
\DeclareMathOperator{\csch}{csch}
\DeclareMathOperator{\arcsinh}{arcsinh}
\DeclareMathOperator{\arccosh}{arccosh}
\DeclareMathOperator{\arctanh}{arctanh}
\DeclareMathOperator{\arccoth}{arccoth}
\DeclareMathOperator{\arcsech}{arcsech}
\DeclareMathOperator{\arccsch}{arccsch}

%% Useful subscripts: opposite, reduced, singularities, etc.
\newcommand{\op}{\textrm{op}}
\newcommand{\red}{\textrm{red}}
\newcommand{\sing}{\textrm{sing}}
\newcommand{\alg}{\textrm{alg}}
\newcommand{\cl}{\textrm{cl}}
\newcommand{\et}{\textrm{\'et}}
\newcommand{\cart}{\textrm{cart}}
\newcommand{\eff}{\textrm{eff}}

%% Various categories
\newcommand{\Set}{\textrm{\textbf{Set}}}
\newcommand{\EqRel}{\textrm{\textbf{EqRel}}}
\newcommand{\Groupoid}{\textrm{\textbf{Groupoid}}}
\newcommand{\Cat}{\textrm{\textbf{Cat}}}
\newcommand{\Ab}{\textrm{\textbf{Ab}}}
\newcommand{\Sch}{\textrm{\textbf{Sch}}}
\newcommand{\Top}{\textrm{\textbf{Top}}}
\newcommand{\hTop}{\textrm{\textbf{hTop}}}
\newcommand{\QCoh}{\textrm{\textbf{QCoh}}}
\newcommand{\QCohAlg}{\textrm{\textbf{QCohAlg}}}
\newcommand{\AffMor}{\textrm{\textbf{AffMor}}}


%% Exterior power
\newcommand{\extp}{\textstyle\bigwedge\nolimits}

%% The typical modulo commands leave too much space before, so we
%% replace them.
% \newcommand{\dmod}{\textrm{-}\textbf{mod}}
\newcommand{\dmod}{\textrm{-mod}}
\renewcommand{\mod}[1]{\ensuremath{\;\textrm{mod }{#1}}}
\renewcommand{\pmod}[1]{\ensuremath{\;(\textrm{mod }{#1})}}


%% Shortcuts
\newcommand{\x}{\ensuremath{\times}}
\newcommand{\wtilde}[1]{\widetilde{#1}}
\newcommand{\what}[1]{\widehat{#1}}
\newcommand{\wcup}{\smallsmile}
\newcommand{\wcap}{\smallfrown}
\newcommand{\cn}{\colon}
\newcommand{\rarr}{\ifinner\rightarrow\else\longrightarrow\fi}
\newcommand{\Rarr}{\ifinner\Rightarrow\else\Longrightarrow\fi}
\newcommand{\larr}{\ifinner\leftarrow\else\longleftarrow\fi}
\newcommand{\Larr}{\ifinner\Leftarrow\else\Longleftarrow\fi}
\newcommand{\hrarr}{\hookrightarrow}
\newcommand{\hlarr}{\hookleftarrow}
\newcommand{\xrarr}{\xrightarrow}
\newcommand{\xlarr}{\xleftarrow}
\newcommand{\drarr}{\dashrightarrow}
\newcommand{\dlarr}{\dashleftarrow}
\newcommand{\acts}{\curvearrowright}
\newcommand{\slashsim}{/\hspace{-4pt}\sim}
% \renewcommand{\/}{/\hspace{-3pt}/}
\newcommand{\groupoid}[4]{
  \ifinner
  {\xymatrix@1@C=15pt{
      {#1} \ar@<3pt>[r]^-{#3} \ar@<-3pt>[r]_-{#4} & {#2}
  }}
  \else
  {\xymatrix{
      {#1} \ar@<3pt>[r]^-{#3} \ar@<-3pt>[r]_-{#4} & {#2}
  }}
  \fi
}


%% Line spacing
% \usepackage{setspace}
% \doublespacing
%% or:
% \onehalfspacing


%% Hypenations

\hyphenation{Groth-en-dieck}

%% Tweaks

\renewcommand{\textrm}{\textnormal}
\renewcommand{\phi}{\varphi}
\renewcommand{\epsilon}{\varepsilon}


%% Useful if typing in a book-like document class.
% \setcounter{tocdepth}{2}
% \numberwithin{subsection}{section}
% \numberwithin{section}{chapter}

%% Use if you want to change the name of a "Chapter". For example, if
%% typing notes it may be useful to replace with "Lecture" instead.
% \renewcommand{\chaptername}{Chapter}


%%% Local Variables: 
%%% mode: latex
%%% TeX-master: "template"
%%% End: 
